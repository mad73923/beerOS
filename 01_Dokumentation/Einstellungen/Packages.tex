% ==========================================================================
% Anpassung Seitenlayout
% ==========================================================================
\usepackage[
	automark,			% Kapitelangaben in Kopfzeile automatisch erstellen
]{scrpage2}

% ==========================================================================
% Anpassung an deutsche Sprache
% ==========================================================================
\usepackage{ngerman}
\usepackage[ngerman]{babel}

% ==========================================================================
% KOMAScriptanpassung (u.a.: Redefine von addtolist)
% ==========================================================================
\usepackage{scrhack}

% ==========================================================================
% f�r compactitem
% ==========================================================================
\usepackage{paralist} 

% ==========================================================================
% Einstellungen f�r �berschriften
% ==========================================================================
\usepackage[bf,big]{titlesec}	% �berschriften schlanker und gr��er %rm,big,compact
\titleformat{\chapter}[display]				% Anpassen des Schriftzugs "Kapitel x"
  {\normalfont\Huge}
  {\chaptertitlename\ \thechapter}{12pt}{\Huge}
  \titlespacing*{\chapter}{0cm}{-\topskip}{30pt}[0pt]
% ==========================================================================
% Einstellung Schriften und Umlaute
% ==========================================================================
\usepackage[ansinew]{inputenc}
\usepackage[T1]{fontenc}
\usepackage{ae} % "sch�neres" �
\usepackage{eurosym}
\usepackage{lmodern}

% ==========================================================================
% SI-Einheiten
% ==========================================================================
\usepackage[locale = DE, binary-units]{siunitx}
\DeclareSIUnit[number-unit-product = {}]\dmips{DMIPS}
\DeclareSIUnit[number-unit-product = {}]\baud{Baud}

\usepackage{icomma}
\usepackage{nicefrac}
% ==========================================================================
% Einbinden von Grafiken
% ==========================================================================
\usepackage[dvips,final]{graphicx}
\graphicspath{{Bilder/}} % Dort liegen die Bilder des Dokuments

% ==========================================================================
% Befehle aus AMSTeX f�r mathematische Symbole z.B. \boldsymbol \mathbb
% ==========================================================================
\usepackage{amsmath,amsfonts}
\usepackage{amsmath, amsthm, amssymb}
\usepackage{mathtools}

% ==========================================================================
% Einfache Definition der Zeilenabst�nde und Seitenr�nder etc.
% ==========================================================================
\usepackage{setspace}
\usepackage{geometry}

% ==========================================================================
% Zum Umflie�en von Bildern
% ==========================================================================
\usepackage{floatflt}
\usepackage{float}    

% ==========================================================================
% Quellcode-Ausgabe
% ==========================================================================
\usepackage{listings}
\usepackage{xcolor} 

% ==========================================================================
% Lange URLs umbrechen etc.
% ==========================================================================
\usepackage[hyphens]{url}

% ==========================================================================
% Einstellungen Bibtex
% ==========================================================================
\usepackage[numbers]{natbib}
\bibliographystyle{alphadin}		%DIN-Stil des Literaturverzeichnisses

% ==========================================================================
% Zum fortlaufenden Durchnummerieren der Fu�noten
% ==========================================================================
\usepackage{chngcntr}

% ==========================================================================
% f�r lange Tabellen
% ==========================================================================
\usepackage{longtable}
\usepackage{array}
\usepackage{ragged2e}
\usepackage{lscape}

% ==========================================================================
% Spaltendefinition rechtsb�ndig mit definierter Breite
% ==========================================================================
\newcolumntype{w}[1]{>{\raggedleft\hspace{0pt}}p{#1}}

% ==========================================================================
% Erweiterung f�r Querverweise
% ==========================================================================
\usepackage[ngerman]{varioref}

% ==========================================================================
% Erweiterung f�r figures
% ==========================================================================
\usepackage[aboveskip=1em,belowskip=0em]{caption}
\usepackage{subfigure}

% ==========================================================================
% Optischer Randausgleich
% ==========================================================================
\usepackage[tracking=true]{microtype}
\DeclareMicrotypeSet*[tracking]{my}{font = */*/*/sc/*}
\SetTracking{ encoding = *, shape = sc }{ 45 }

% ==========================================================================
% Auswahl der Schriftart
% ==========================================================================
% Arial mit euler f�r Mathe:
%\usepackage[scaled=0.92]{uarial}			% Paket f�r Schrift Arial 
%\renewcommand\familydefault{\sfdefault} % Umstellen auf serifenlose Schrift
%\usepackage{eulervm}

% Times mit passendem Mathe-Font
\renewcommand{\rmdefault}{ptm}
\usepackage{times}
\usepackage{txfonts}
% Kapitel im Inhaltsverzeichnis auch in Times darestellen
\setkomafont{chapterentry}{\rmfamily\bfseries\normalsize}	

% ==========================================================================
% griechische Buchstaben
% ==========================================================================
\usepackage{upgreek}	

% ==========================================================================
% Mehrspaltiger Text
% ==========================================================================
\usepackage{multicol}
\columnseprule 1pt	% Linie swischen Spalten
\columnsep 15mm 	% Abstand zwischen Spaletn

% ==========================================================================
% ToDo-List
% ==========================================================================
\usepackage[
%disable,   %hier wird die ToDo List deaktiviert!!!
ngerman,
textsize=footnotesize,
colorinlistoftodos,
figwidth=0.8\textwidth
]{todonotes}

% ==========================================================================
% Draft Markierung
% ==========================================================================
\usepackage[
final, 			%Aktivierung durch draft; Deaktivierung durch final
%draft,
allpages,
color=gray,
mark=\mbox{vorl�ufiger Entwurf zur Durchsicht},
angle=90,
scale=0.08,
xcoord=-96,
ycoord=-35
]{draftmark}

% ==========================================================================
% Sonstiges
% ==========================================================================
\usepackage{pdflscape}
\usepackage{wallpaper}
\usepackage{ifthen}												% Bedingungen

\setlength{\marginparwidth}{2cm}
\reversemarginpar

% ==========================================================================
% Glossar, Abk�rzungsverzeichnis, Symbolverzeichnis
% ==========================================================================
\usepackage[
nonumberlist, %keine Seitenzahlen anzeigen
acronym,      %ein Abk�rzungsverzeichnis erstellen
toc]      %im Inhaltsverzeichnis auf section-Ebene erscheinen
{glossaries}

% Eigene Definition Abk�rzungsverzeichnis:
\newglossarystyle{myacro}
{
	\renewenvironment{theglossary}{}{}
	\renewcommand*{\glossaryheader}{}
	\renewcommand*{\glsgroupheading}[1]{}
	\renewcommand*{\glsgroupskip}{}
	\renewcommand*{\glossaryentryfield}[5]
	{
		\begin{minipage}[t]{0.2\textwidth}
			\glstarget{##1}{##2}~\dotfill
		\end{minipage}
		\begin{minipage}[t]{0.7\textwidth}
			~##3
		\end{minipage}
		\vskip -0.16cm  % Abstand zwischen den Zeilen
	}
	\renewcommand*{\glossarysubentryfield}[6]{%
	\glossaryentryfield{##2}{##3}{##4}{##5}{##6}}%
}

% Eigene Definition Symbolverzeichnis:
\newglossarystyle{mysombol}
{
	\renewenvironment{theglossary}
	{\begin{multicols}{2}\noindent}{\end{multicols}}
	\renewcommand*{\glossaryheader}{}
	\renewcommand*{\glsgroupheading}[1]{}
	\renewcommand*{\glsgroupskip}{}
	\renewcommand*{\glossaryentryfield}[3]
	{\noindent
		\begin{minipage}[t]{0.94\columnwidth}
			\begin{minipage}[t]{0.2\columnwidth}
				\glstarget{##1}{##2}
			\end{minipage}
			\begin{minipage}[t]{0.78\columnwidth}
				##3
			\end{minipage}
		\end{minipage}
		\vskip 0.01em   % Abstand zwischen den Zeilen
	}
	\renewcommand*{\glossarysubentryfield}[3]{%
	\glossaryentryfield{##2}{##3}}%
}
%Ein eigenes Symbolverzeichnis erstellen
\newglossary[slg]{symbolslist}{syi}{syg}{Symbolverzeichnis}
%Den Punkt am Ende jeder Beschreibung deaktivieren
\renewcommand*{\glspostdescription}{}

% ==========================================================================
% PDF-Optionen (Hyperref) (immer zuletzt)
% ==========================================================================
\usepackage[
%colorlinks=false,
bookmarks,
bookmarksopen=true,
pdftitle={\art},
pdfauthor={\autor},
pdfcreator={\autor},
pdfsubject={\titel},
pdfkeywords={\titel},
plainpages=false,% zur korrekten Erstellung der Bookmarks
pdfpagelabels=true,% zur korrekten Erstellung der Bookmarks
hypertexnames=false,% zur korrekten Erstellung der Bookmarks
breaklinks=true
%linktocpage % Seitenzahlen anstatt Text im Inhaltsverzeichnis verlinken
]{hyperref}

\usepackage{bookmark}