% ==========================================================================
% Formatvorlage f�r Diplomarbeiten der HS Karlsruhe und Furtwangen
% Erstellt von Stefan Jassen, Stefan R�mmele-Werner und Manuel Sautermeister
% ==========================================================================

%==================================================================================
% Themeauswahl
% htwg    - HTWG Konstanz
%==================================================================================
\newcommand{\layoutpres}{htwg}

% ==========================================================================
% Meta-Informationen
% Informationen �ber das Dokument, wie z.B. Titel, Autor, Matrikelnr. etc
% ==========================================================================
% Informationen ------------------------------------------------------------
% 	Definition von globalen Parametern, die im gesamten Dokument verwendet
% 	werden k�nnen (z.B auf dem Deckblatt etc.).
% --------------------------------------------------------------------------
\newcommand{\titel}{Angewandte Systemprogrammierung}
\newcommand{\art}{AVR OS}
\newcommand{\Fakultaet}{Name der aktuellen Fakult�t}
\newcommand{\autor}{Andreas Reinhardt}
\newcommand{\autors}{Matthias Weis}
\newcommand{\autorsmall}{Andreas Reinhardt, Matthias Weis}
\newcommand{\Studiengang}{Name des aktuellen Studiengangs}
\newcommand{\matrikelnr}{123456}
\newcommand{\betreuer}{Daniel Urbanietz}
\newcommand{\referent}{Prof.\,Dr. Erstkorrektor}
\newcommand{\korreferent}{Prof.\,Dr. Zweitkorrektor}
\newcommand{\bearbeitungszeit}{Wintersemester 15/16}
\newcommand{\company}{\mbox{XY}}

% Eigene Befehle
\newcommand{\matlab}[1]{\lstinline[basicstyle=\ttfamily\normalsize]|#1|}
\newcommand{\tief}[1]{$_\text{#1}$}			% Tiefstellen
\newcommand{\hoch}[1]{$^\text{#1}$}			% Hochstellen


% Befehle F�r Abk�rzungs- und Symbolverzeichnis\gls*{symb:Lambda}
\newcommand{\symb}[1]{\glslink*{#1}{\gls*{#1}}}	% Gibt Symbol aus
\newcommand{\abk}[1]{\glslink*{#1}{#1}}	% Gibt Abk�rzung aus
\newcommand{\addabk}[2]{\newacronym{#1}{#1}{#2}}	% F�gt Abk�rzung hinzu (nur Pr�ambel)

\newcommand{\addsymb}[3] 	% F�gt Symbol hinzu (nur Pr�ambel)
{
	\newglossaryentry{#1}
	{
		name=#2,
		description={#3},
		sort=symbolphi, type=symbolslist
	}
}

% Befehle zur semantischen Auszeichnung von Text
\newcommand{\Fachbegriff}[1]{\textit{#1}}

% weitere Abk�rzungen
\newcommand{\ua}{\mbox{u.\,a.\ }}
\newcommand{\zb}{\mbox{z.\,B.\ }}
\newcommand{\bs}{$\backslash$}



% ==========================================================================
% Dokumentenkopf
% ==========================================================================
\documentclass[
	%12pt,								% Schriftgr��e kann hier ge�ndert werden
	%DIV=calc,						% Neuberechnung des Satzspiegels		
	ngerman,							% f�r Umlaute, Silbentrennung etc.
	titlepage,						% es wird eine Titelseite verwendet
	a4paper,							% Papierformat
	%headings=normal,			% Schriftgr��e in �berschriften
	numbers=noenddot,			% Nummerierung der �berschrift erh�lt keinen Endpunkt
	listof=totoc,					% Verzeichnisse im Inhaltsverzeichnis auff�hren
	bibliography=totoc,		% Literaturverzeichnis im Inhaltsverzeichnis auff�hren
	index=totoc,					% Index im Inhaltsverzeichnis auff�hren
	parskip=half,					% Absatzabstand (half/full)
	oneside,							% Druck: Einseitig: oneside; Doppelseitig: twoside
												% (passt Rand und Kopf-/Fu�zeile an)
	chapterprefix=true		% "Kapitel 1" statt "1. Kapitel"
	]{scrreprt}
	
% ==========================================================================
% Ben�tigte Packages
% ==========================================================================
% ==========================================================================
% Anpassung Seitenlayout
% ==========================================================================
\usepackage[
	automark,			% Kapitelangaben in Kopfzeile automatisch erstellen
]{scrpage2}

% ==========================================================================
% Anpassung an deutsche Sprache
% ==========================================================================
\usepackage{ngerman}
\usepackage[ngerman]{babel}

% ==========================================================================
% KOMAScriptanpassung (u.a.: Redefine von addtolist)
% ==========================================================================
\usepackage{scrhack}

% ==========================================================================
% f�r compactitem
% ==========================================================================
\usepackage{paralist} 

% ==========================================================================
% Einstellungen f�r �berschriften
% ==========================================================================
\usepackage[bf,big]{titlesec}	% �berschriften schlanker und gr��er %rm,big,compact
\titleformat{\chapter}[display]				% Anpassen des Schriftzugs "Kapitel x"
  {\normalfont\Huge}
  {\chaptertitlename\ \thechapter}{12pt}{\Huge}
  \titlespacing*{\chapter}{0cm}{-\topskip}{30pt}[0pt]
% ==========================================================================
% Einstellung Schriften und Umlaute
% ==========================================================================
\usepackage[ansinew]{inputenc}
\usepackage[T1]{fontenc}
\usepackage{ae} % "sch�neres" �
\usepackage{eurosym}
\usepackage{lmodern}

% ==========================================================================
% SI-Einheiten
% ==========================================================================
\usepackage[locale = DE, binary-units]{siunitx}
\DeclareSIUnit[number-unit-product = {}]\dmips{DMIPS}
\DeclareSIUnit[number-unit-product = {}]\baud{Baud}

\usepackage{icomma}
\usepackage{nicefrac}
% ==========================================================================
% Einbinden von Grafiken
% ==========================================================================
\usepackage[dvips,final]{graphicx}
\graphicspath{{Bilder/}} % Dort liegen die Bilder des Dokuments

% ==========================================================================
% Befehle aus AMSTeX f�r mathematische Symbole z.B. \boldsymbol \mathbb
% ==========================================================================
\usepackage{amsmath,amsfonts}
\usepackage{amsmath, amsthm, amssymb}
\usepackage{mathtools}

% ==========================================================================
% Einfache Definition der Zeilenabst�nde und Seitenr�nder etc.
% ==========================================================================
\usepackage{setspace}
\usepackage{geometry}

% ==========================================================================
% Zum Umflie�en von Bildern
% ==========================================================================
\usepackage{floatflt}
\usepackage{float}    

% ==========================================================================
% Quellcode-Ausgabe
% ==========================================================================
\usepackage{listings}
\usepackage{xcolor} 

% ==========================================================================
% Lange URLs umbrechen etc.
% ==========================================================================
\usepackage[hyphens]{url}

% ==========================================================================
% Einstellungen Bibtex
% ==========================================================================
\usepackage[numbers]{natbib}
\bibliographystyle{alphadin}		%DIN-Stil des Literaturverzeichnisses

% ==========================================================================
% Zum fortlaufenden Durchnummerieren der Fu�noten
% ==========================================================================
\usepackage{chngcntr}

% ==========================================================================
% f�r lange Tabellen
% ==========================================================================
\usepackage{longtable}
\usepackage{array}
\usepackage{ragged2e}
\usepackage{lscape}

% ==========================================================================
% Spaltendefinition rechtsb�ndig mit definierter Breite
% ==========================================================================
\newcolumntype{w}[1]{>{\raggedleft\hspace{0pt}}p{#1}}

% ==========================================================================
% Erweiterung f�r Querverweise
% ==========================================================================
\usepackage[ngerman]{varioref}

% ==========================================================================
% Erweiterung f�r figures
% ==========================================================================
\usepackage[aboveskip=1em,belowskip=0em]{caption}
\usepackage{subfigure}

% ==========================================================================
% Optischer Randausgleich
% ==========================================================================
\usepackage[tracking=true]{microtype}
\DeclareMicrotypeSet*[tracking]{my}{font = */*/*/sc/*}
\SetTracking{ encoding = *, shape = sc }{ 45 }

% ==========================================================================
% Auswahl der Schriftart
% ==========================================================================
% Arial mit euler f�r Mathe:
%\usepackage[scaled=0.92]{uarial}			% Paket f�r Schrift Arial 
%\renewcommand\familydefault{\sfdefault} % Umstellen auf serifenlose Schrift
%\usepackage{eulervm}

% Times mit passendem Mathe-Font
\renewcommand{\rmdefault}{ptm}
\usepackage{times}
\usepackage{txfonts}
% Kapitel im Inhaltsverzeichnis auch in Times darestellen
\setkomafont{chapterentry}{\rmfamily\bfseries\normalsize}	

% ==========================================================================
% griechische Buchstaben
% ==========================================================================
\usepackage{upgreek}	

% ==========================================================================
% Mehrspaltiger Text
% ==========================================================================
\usepackage{multicol}
\columnseprule 1pt	% Linie swischen Spalten
\columnsep 15mm 	% Abstand zwischen Spaletn

% ==========================================================================
% ToDo-List
% ==========================================================================
\usepackage[
%disable,   %hier wird die ToDo List deaktiviert!!!
ngerman,
textsize=footnotesize,
colorinlistoftodos,
figwidth=0.8\textwidth
]{todonotes}

% ==========================================================================
% Draft Markierung
% ==========================================================================
\usepackage[
final, 			%Aktivierung durch draft; Deaktivierung durch final
%draft,
allpages,
color=gray,
mark=\mbox{vorl�ufiger Entwurf zur Durchsicht},
angle=90,
scale=0.08,
xcoord=-96,
ycoord=-35
]{draftmark}

% ==========================================================================
% Sonstiges
% ==========================================================================
\usepackage{pdflscape}
\usepackage{wallpaper}
\usepackage{ifthen}												% Bedingungen

\setlength{\marginparwidth}{2cm}
\reversemarginpar

% ==========================================================================
% Glossar, Abk�rzungsverzeichnis, Symbolverzeichnis
% ==========================================================================
\usepackage[
nonumberlist, %keine Seitenzahlen anzeigen
acronym,      %ein Abk�rzungsverzeichnis erstellen
toc]      %im Inhaltsverzeichnis auf section-Ebene erscheinen
{glossaries}

% Eigene Definition Abk�rzungsverzeichnis:
\newglossarystyle{myacro}
{
	\renewenvironment{theglossary}{}{}
	\renewcommand*{\glossaryheader}{}
	\renewcommand*{\glsgroupheading}[1]{}
	\renewcommand*{\glsgroupskip}{}
	\renewcommand*{\glossaryentryfield}[5]
	{
		\begin{minipage}[t]{0.2\textwidth}
			\glstarget{##1}{##2}~\dotfill
		\end{minipage}
		\begin{minipage}[t]{0.7\textwidth}
			~##3
		\end{minipage}
		\vskip -0.16cm  % Abstand zwischen den Zeilen
	}
	\renewcommand*{\glossarysubentryfield}[6]{%
	\glossaryentryfield{##2}{##3}{##4}{##5}{##6}}%
}

% Eigene Definition Symbolverzeichnis:
\newglossarystyle{mysombol}
{
	\renewenvironment{theglossary}
	{\begin{multicols}{2}\noindent}{\end{multicols}}
	\renewcommand*{\glossaryheader}{}
	\renewcommand*{\glsgroupheading}[1]{}
	\renewcommand*{\glsgroupskip}{}
	\renewcommand*{\glossaryentryfield}[3]
	{\noindent
		\begin{minipage}[t]{0.94\columnwidth}
			\begin{minipage}[t]{0.2\columnwidth}
				\glstarget{##1}{##2}
			\end{minipage}
			\begin{minipage}[t]{0.78\columnwidth}
				##3
			\end{minipage}
		\end{minipage}
		\vskip 0.01em   % Abstand zwischen den Zeilen
	}
	\renewcommand*{\glossarysubentryfield}[3]{%
	\glossaryentryfield{##2}{##3}}%
}
%Ein eigenes Symbolverzeichnis erstellen
\newglossary[slg]{symbolslist}{syi}{syg}{Symbolverzeichnis}
%Den Punkt am Ende jeder Beschreibung deaktivieren
\renewcommand*{\glspostdescription}{}

% ==========================================================================
% PDF-Optionen (Hyperref) (immer zuletzt)
% ==========================================================================
\usepackage[
%colorlinks=false,
bookmarks,
bookmarksopen=true,
pdftitle={\art},
pdfauthor={\autor},
pdfcreator={\autor},
pdfsubject={\titel},
pdfkeywords={\titel},
plainpages=false,% zur korrekten Erstellung der Bookmarks
pdfpagelabels=true,% zur korrekten Erstellung der Bookmarks
hypertexnames=false,% zur korrekten Erstellung der Bookmarks
breaklinks=true
%linktocpage % Seitenzahlen anstatt Text im Inhaltsverzeichnis verlinken
]{hyperref}

\usepackage{bookmark}

% ==========================================================================
% Seiteneinstellungen
% ==========================================================================
% ==========================================================================
% Zeilenabstand
% ==========================================================================
\onehalfspacing

% ==========================================================================
% Farbdefinitionen
% ==========================================================================
\definecolor{grau}{RGB}{240,240,240} %Beispielwert f�r Grau

% ==========================================================================
% Seitenr�nder
% ==========================================================================
\geometry{headsep=0.6cm}				% Abstand Text Oberkante zu Kopfzeile Unterkante
\geometry{footskip=1.1cm}				% Abstand Text Unterkante zu Fu�zeile Oberkante
\geometry{top=3.2cm}						% Abstand Text zu oberem Papierrand
\geometry{bottom=3.7cm}					% Abstand Text zu unterem Papierrand
\geometry{left=3.45cm}					% Abstand Text zu linkem Papierrand
\geometry{right=2.7cm}					% Abstand Text zu rechtem Papierrand

\setlength{\headheight}{1.1\baselineskip}

% ==========================================================================
% Einstellungen Kopf- und Fu�zeile
% ==========================================================================
\pagestyle{scrheadings}
\renewcommand*{\chapterpagestyle}{scrheadings} % Kopf-/Fu�zeile  auf Kapitelanfang
\renewcommand{\headfont}{\normalfont} % Schriftform der Kopfzeile
% Kopfzeile
\ihead[]{\upshape\headmark}
\chead[]{}
\ohead[]{}
\setheadsepline{0.5pt}
% Fu�zeile
\ifthenelse{\equal{\layoutpres}{htwg}}
{
	\refoot[]{\raisebox{-1.0cm}{\includegraphics[height=0.9cm]{Bilder/HTWG_Logo}}}
}{}
\cefoot[]{\raisebox{-0.1cm}{\footnotesize{\upshape\art~\autorsmall}}}

\cofoot[]{\raisebox{-0.1cm}{\footnotesize{\upshape\art~\autorsmall}}}

\rofoot[]{\upshape\pagemark}

\ifthenelse{\equal{\layoutpres}{htwg}}
{
	\lofoot[]{\raisebox{-1.0cm}{\includegraphics[height=0.9cm]{Bilder/HTWG_Logo}}}
}{}
\setfootsepline{0.5pt}

% ==========================================================================
% erzeugt ein wenig mehr Platz hinter einem Punkt
% ==========================================================================
\frenchspacing 

% ==========================================================================
% Nummerierungstiefe ausw�hlen
% ==========================================================================
\setcounter{secnumdepth}{3} % Bis 1.1.1.1
\setcounter{tocdepth}{4} % Bis 1.1.1.1

% ==========================================================================
% Schusterjungen und Hurenkinder vermeiden
% ==========================================================================
\clubpenalty = 10000
\widowpenalty = 10000 
\displaywidowpenalty = 10000

% ==========================================================================
% Definition von Farben f�r Quellcodeausgabe
% ==========================================================================
\definecolor{hellgelb}{rgb}{1,1,0.9}
\definecolor{quellcode_comment}{rgb}{0.2,0.6,0.2}
\definecolor{quellcode_string}{rgb}{0.6,0,0.8}
\definecolor{quellcode_background}{rgb}{0.98,0.98,0.98}
\lstset{numbers=left,
				backgroundcolor=\color{hellgelb},
				breaklines=true,
				frame=single,
				captionpos=b,
 				language=C,
 				basicstyle=\ttfamily\color{black}\scriptsize,
 				keywordstyle=\color{blue},
 				commentstyle=\color{quellcode_comment},
 				stringstyle=\color{quellcode_string}}
 				
\renewcommand{\lstlistingname}{Code} 				

% ==========================================================================
% Fu�noten fortlaufend durchnummerieren
% ==========================================================================
\counterwithout{footnote}{chapter}

% ==========================================================================
% Definitionen des GLossars
% ==========================================================================
%Verwendete Abkürzungen

\addabk{ASIC}{\textbf{A}pplication \textbf{S}pecific \textbf{I}ntegrated \textbf{C}ircuit}
\addabk{SIL}{\textbf{S}afety \textbf{I}ntegrity \textbf{L}evel}
\addabk{UART}{\textbf{U}niversal \textbf{A}synchronous \textbf{R}eceiver/\textbf{T}ransmitter}
\addabk{SPI}{\textbf{S}erial \textbf{P}eriphal \textbf{I}nterface}
\addabk{MISO}{\textbf{M}aster \textbf{I}n \textbf{S}lave \textbf{O}ut}
\addabk{MOSI}{\textbf{M}aster \textbf{O}ut \textbf{S}lave \textbf{I}n}
\addabk{ADC}{\textbf{A}nalog \textbf{D}igital \textbf{C}onverter}
\addabk{DAC}{\textbf{D}igital \textbf{A}nalog \textbf{C}onverter}
\addabk{DSP}{\textbf{D}igitaler \textbf{S}ignal\textbf{p}rozessor}
\addabk{CRC}{\textbf{C}yclic \textbf{R}edundancy \textbf{C}heck}
\addabk{CPU}{\textbf{C}entral \textbf{P}rocessing \textbf{U}nit}
\addabk{DMA}{\textbf{D}irect \textbf{M}emory \textbf{A}ccess}
\addabk{DTC}{\textbf{D}ata \textbf{T}ransfer \textbf{C}ontroller}
\addabk{MIPS}{\textbf{M}illion \textbf{I}nstructions \textbf{P}er \textbf{S}econd}
\addabk{DMIPS}{\textbf{D}hrystone \textbf{MIPS}}
\addabk{RAM}{\textbf{R}andom \textbf{A}ccess \textbf{M}emory}
\addabk{EMV}{\textbf{E}lectro\textbf{m}agnetische \textbf{V}ertr�glichkeit}
\addabk{PAP}{\textbf{P}rogramm\textbf{a}blauf\textbf{p}lan}
\addabk{SCI}{\textbf{S}erial \textbf{C}ommunication \textbf{I}nterface}
\addabk{USART}{\textbf{U}niversial \textbf{S}ynchronous/\textbf{A}synchronous \textbf{R}eceiver/\textbf{T}ransmitter}
\addabk{HART}{\textbf{H}ighway \textbf{A}ddressable \textbf{R}emote \textbf{T}ransducer}

\addabk{ISR}{Interrupt Service Routine}
\addabk{TCB}{Task-Controll-Block}
%Symbole
\addsymb{Rv}{\ensuremath{R_\text{v}}}{Vorwiderstand}
\addsymb{f}{\ensuremath{f}}{Frequenz}

\makeglossaries				% Glossar erstellen

\begin{document}

% ==========================================================================
% Definitionen f�r Silbentrennung (Fehlerhafte Trennung von LaTex)
% ==========================================================================
%fuer ggf. falsche Silbentrennungen
\hyphenation{Trenn-bar-es}
\hyphenation{Mess-un-ge-nau-ig-keit}
\hyphenation{mehr-er-er}
\hyphenation{von-ei-nan-der}


% ==========================================================================
% Deckblatt (Auswahl aus zwei verschiedenen)
% ==========================================================================
\pagenumbering{alph}		% Deckblatt erh�llt Seitennummer a, damit Seite 1 nicht doppelt belegt ist 

%\newgeometry{left=1.8cm,right=0.9cm,bottom=1.3cm,top=1.5cm}
\thispagestyle{plain}
\ThisTileWallPaper{\paperwidth}{\paperheight}{Cover/DeckblattAusPDF.pdf}
\begin{titlepage}	
	\hfill
\end{titlepage}
\restoregeometry
 %Deckblatt aus PDF "DeckblattAusPDF.pdf" in Ordner Cover
%\cleardoublepage

\newgeometry{left=1.8cm,right=0.9cm,bottom=1.3cm,top=1.5cm}
\thispagestyle{plain}
\ifthenelse{\equal{\layoutpres}{htwg}}
{
	\ThisTileWallPaper{\paperwidth}{\paperheight}{Bilder/Cover_HTWG.pdf}
}{}
\begin{titlepage}	
		\sffamily
			\begin{minipage}[t!]{0.35\textwidth}
			\begin{flushleft}
				\ifthenelse{\equal{\layoutpres}{htwg}}
				{
					\includegraphics[width=0.75\textwidth]{Bilder/HTWG_Logo}
				}{}
			\end{flushleft}
			\end{minipage}
			\hfill
			\begin{minipage}[t!]{0.5\textwidth}
			\begin{flushright}
			
			\end{flushright}
			\end{minipage}
			%\vfill
			\vskip 1cm
			\hskip 8.5cm
		\begin{minipage}[t!]{0.57\textwidth}
			\begin{flushleft}
				\hfill
				\vskip 2cm
				{\fontsize{45}{10}\textsc{\art}}
				\vskip 2.7cm
				{\Huge\titel\\}
			\end{flushleft}
		\end{minipage}
			\vfill

			\begin{minipage}[!b]{0.38\textwidth}
			\hfill
			\end{minipage}
			\quad
			{\normalsize
			\begin{minipage}[!b]{0.57\textwidth}
				\begin{minipage}[t]{0.38\textwidth}
						Name:\\
						\\
						Betreuer:\\
						Bearbeitungszeit:\\
				\end{minipage}
				\begin{minipage}[t]{0.57\textwidth}
					\autor\\
					\autors\\
					\betreuer\\
					\bearbeitungszeit
				\end{minipage}
			\end{minipage}
			}
%		\end{addmargin}
	\ClearWallPaper
\end{titlepage}
\restoregeometry
 %normales Deckblatt
\cleardoublepage

% ==========================================================================
% Inhaltsverzeichnis
% ==========================================================================
\pagenumbering{Roman}		% Seitennummer in r�mischen Ziffern
\chapter*{Abstract}
Im Rahmen der Lehrveranstaltung angewandte Systemprogrammierung
wurde im Wintersemester 15/16 das hier beschriebene Betriebssystem f�r Microcontroller entwickelt.
Au�er den funktionalen Vorgaben gab es keine weiteren Bedingungen
oder vorgefertigte Codest�cke.
F�r das minimalistische Betriebssystem wurden Module wie der
Dispatcher, Scheduler und Synchronisierungsmechanismen unter
Betracht gezogen. Zudem sollten zwei zus�tliche, frei w�hlbare 
Features implementiert werden. Am Ende stand ein agiles, 
multitasking-f�higes Betriebssystem.
\tableofcontents			% Inhaltsverzeichnis einf�gen
\cleardoublepage

% ==========================================================================
% Inhalt (Einbinden der Dokumente in Ordner Inhalt)
% ==========================================================================
\pagenumbering{arabic} % Seitennummer in arabische Ziffern

\chapter{System�bersicht}

F�r ein minimalistisches Betriebssystem werden einige Grundbausteine
ben�tigt. Zun�chst muss die Hardware bzw. der Chip initialisiert
werden, indem verschiedene Zeitquellen, Timer und die globale
Interruptverwaltung konfiguriert werden. \todo{verweis} Bevor
das eigentliche Betriebssystem gestartet werden kann, m�ssen zudem
diverse Strukturen f�r die Tasks aufgestellt werden, sodass diese
unabh�ngig voneinander lauff�hig sind. \todo{verweis}
Zum Starten des Systems wird mittels eines modifizierten Aufrufs
der Dispatcher erstmalig durchlaufen und die erste Task zum Ausf�hren
vorbereitet. Dabei wird auch der Scheduler aufgerufen, der, abh�ngig vom
ausgew�hlten Algorithmus, die n�chste Task bestimmt.\todo{verweis}

Die soeben genannten Module w�rden f�r ein multitasking-f�higes
Betriebssystem bereits ausreichen. Jedoch w�re sein Funktionsumfang
sehr beschr�nkt. Bislang besteht noch keine M�glichkeit einer sicheren
Interaktion zwischen zwei oder mehreren Tasks. Hierf�r werden 
z.B. kritische Bereiche oder Signale ben�tigt, welche das Modul
Synchronisation bereitstellt. Die Anwendung von Wartezeiten innerhalb
Tasks setzt eine Systemzeit vorraus. Das Modul Zeiten \todo{verweis} verwendet hierf�r Trigger der Hardware.
Eine gro�e Erleichterung im Entwicklungsprozess (und auch 
im sp�teren Anwendungsfall) ist
die Verwendung von Collections, welche Datenpakete hantierbar 
macht. Gleichzeitig verbessert sich mit ihnen die Lesbarkeit des
Programmcodes.

\section{Hardware}
Als Zielplattform wurde die weit verbreitete AVR-Serie von Atmel
verwendet. Jedoch wurde w�hrend der Umsetzung darauf geachtet, 
dass bei einem Plattformwechsel m�glichst wenig Code angepasst
werden muss. Als Entwicklungstool wurde der AVR-Simulator verwendet,
welcher innherhalb des AVR-Studios von Atmel bereitgestellt wird.
Als Zielbaustein wurde der \mbox{ATxmega128A1} gew�hlt.

Die minimale Vorraussetzung an die Hardware ist ein Timer
und ein Interruptsystem.
Die Periode des Timers muss bekannt oder definierbar sein.
Sind neben dem als Systemtimer verwendeten Timer noch weitere Timer
verf�gbar (wie das beim \mbox{ATxmega128A1} der Fall ist), k�nnen diese vom Entwickler\todo{welcher entwickler, vom OS? auf OS?}frei verwendet werden, 
unter der Bedingung, dass der Systemtimer nicht beeinflusst 
wird.

\subsection{Zeitquellen}
Zuallererst wird der interne Oszillator auf eine Systemfrequenz von $f_{S} = 32 \text{MHz}$ konfiguriert. Dadurch k�nnen durch
nachgeschaltete Teiler diverse Frequenzen an weiterer
Peripherie erzeugt werden \todo{verweis}.

\subsection{Dispatcher Timer}
Damit nach Ablauf der Zeitscheibe einer Task der Dispatcher 
aufgerufen wird, muss ein Dispatcher Timer initialisiert werden.
Dieser l�st, unabh�ngig vom Ausf�hrungszustand\todo{verweis}
\footnote{Es sei denn, die Task befindet sich in einem kritischen 
(Interrupt gesperrten) Abschnitt} der aktuellen Task, einen 
Interrupt aus, in dessen \gls{ISR} der Dispatcher aufgerufen
wird. Au�erdem dient der Timer zur Berechnung der aktuellen
Systemzeit.

\begin{lstlisting}[language=C, label=lst:DispatcherTimer, caption=Initialisierung Dispatcher Timer]
void initDispatcherTimer(){
	TCF0.CTRLB = TC_WGMODE_NORMAL_gc;
	TCF0.PER = 0x7D00;
	TCF0.INTCTRLA = TC_OVFINTLVL_HI_gc;
}
\end{lstlisting}

In dieser Konfiguration z�hlt der Timer aufw�rts bis zum Wert der
Periode (\texttt{TCF0.PER}) und wird auf 0 zur�ckgesetzt. 
Gleichzeitig l�st er beim Zur�cksetzen einen �berlauf-Interrupt aus.

Wird als Zeitquelle die Systemfrequenz $f_{S}$ ohne Vorteiler
gew�hlt, so ergibt sich folgendes Intervall:
\begin{equation}
\begin{split}
t_{per} = \frac{7\text{D}00_{16}}{32\text{MHz}} = 1\text{ms}
 \end{split}
 \label{eq:intervall}
\end{equation}

In der Annahme, dass keine kritische Abschnitte verwendet werden
wird somit in Abst�nden von $t_{per}$ ein Kontextwechsel herbeigef�hrt. Sollten kritische Abschnitte\todo{kritische abschnitte oder interruptsperren, comment in file} %eine mutex ist doch auch ein kritischer abschnitt, bei einer mutex wird aber kein interrupt verloren oder?, es werden ja nur beim eintreten und verlassen des kritischen abschnitts die interrupts gesperrt  
Anwendung finden liegt
es am Entwickler, diese m�glichst kurz zu halten. Solange sie
deutlich unter $t_{per}$ liegen gibt es keine Probleme. Bedenklich 
wird es, wenn die Ausf�hrung des kritischen Bereichs l�nger als 
$t_{per}$ dauert,
da hierbei ein oder im schlimmsten Fall gleich mehrere 
Interruptsignale verloren gehen und dadurch u.a. die Systemzeit
nicht mehr pr�zise ist.

\subsection{Interrupts}
Auch die Konfiguration des Interruptsystems ist nicht besonders
aufw�ndig. Auf dem \mbox{ATxmega128A1} muss jediglich das
dem Dispatcher Timer entsprechende Interuptlevel scharf gestellt
und das globale Interruptenable gesetzt werden.

Da die Funktionen zum L�schen oder Setzen des Interrupts auch
mit dem Betreten oder Verlassen eines kritischen Abschnitts
gleichzusetzen sind, werden sie plattformunabh�ngig exportiert.
Wichtig hierbei ist nur, dass die Funktionen inline aufgerufen
werden, sodass im Anwendungsfall der Stack nicht manipuliert wird.
Dies wird beim AVR-Compiler durch das Attribut 
\texttt{always\_inline} verwirklicht.
\begin{lstlisting}[language=C, label=lst:interruptDisEnable, caption=Export der Interruptfunktionen]
static void __attribute__((always_inline)) enableInterrupts(){
	sei();
}
static void __attribute__((always_inline)) disableInterrupts(){
	cli();
}
\end{lstlisting}

\section{Strukturen}
In einem multitaskingf�higen Betriebssystem m�ssen neben dem
Stack f�r jede Task auch ein sogenannter \gls{TCB} 
initialisiert werden. Dieser enth�lt Informationen
�ber den aktuellen Zustand, die Priorit�t\todo{verweis scheduler},
den Stack-/pointer und Zeitangaben \todo{verweis schlafen}.

\subsection{Task-Control-Block}
Als m�gliche Zust�nde einer Task wurden wie �blich 
\texttt{READY}, \texttt{RUNNING}, \texttt{WAITING} und 
\texttt{KILLED} definiert. Bei der Priorit�t bedeutet
eine kleinere Zahl einen h�heren Vorrang (mit 0 als
h�chsten Wert).
\begin{lstlisting}[language=C, label=lst:TCB, caption=Task-Control-Block]
typedef volatile struct strucTCB{
	volatile uint8_t prio;
	volatile uint8_t id;
	volatile uint8_t* stackPointer;
	volatile uint8_t* stackBeginn;
	volatile uint16_t stackSize;
	volatile uint32_t waitUntil;
	volatile uint8_t tmpPrio;
	volatile taskstate state;
}taskControlBlock;
\end{lstlisting}

\subsection{Stack und Stackpointer}
Zur Initialisierung eines \gls{TCB} geh�rt auch der eigentliche
Stack sowie dessen Zeiger. Daher muss vor dem Start des
Betriebsystems auch der Speicherbereich des Stacks reserviert
und vorbereitet werden.

Zur Vorbereitung eines Stacks wird an seiner ersten und letzten 
Speicherstelle eine Magicnumber\footnote{z.B. AA$_{16} = 
10101010_{2}$} eingef�gt. Mit dieser
soll erkannt werden, ob es zu einem Stack�berlauf gekommen ist.
\todo{verweis dispatcher} Zudem wird an die vorletzte Stelle
die R�cksprungadresse des Programmcodes kopiert, welcher in der
jeweiligen Task ausgef�hrt werden soll. Der Task-Stackpointer
wird daraufhin auf die Adresse x stellen �berhalb der letzten
Adresse gesetzt. Das x entspricht hierbei der Anzahl der
CPU-Register.\todo{verweis dispatcher}

Zum Starten des Systems wird bei der Starttask der Stackpointer
nochmals ver�ndert: Um gleich zur R�cksprungadresse bzw. 
Startadresse des Task-Programmcodes zu gelangen, wird der
Stackpointer direkt �berhalb dieser Adresse gesetzt und
mittels \texttt{RET}-Befehl zu selbiger gesprungen.
\todo{naming beer??}
\begin{lstlisting}[language=C, label=lst:systemstart, caption=Systemstart]
void beerOS_start(taskControlBlock* firstTask, void (*scheduler_init)(void)){
	currentTask = firstTask;
	scheduler_init();
	time_init();
	firstTask->state = RUNNING;
	mainSP = SP;
	//set stack pointer of starting task next to taskaddress
	SP = firstTask->stackBeginn+firstTask->stackSize-progcntOffset;
	//start task
	asm volatile ("ret");
}
\end{lstlisting}

\section{Dispatcher}
Im sp�teren Betrieb soll es f�r jede Tasks so aussehen, als w�re 
sie die einzige im System. D.h. wird eine Task unterbrochen,
m�ssen nicht nur alle Register und der CPU-Zustand gesichert,
sondern auch der Stackpointer gespeichert und versetzt
werden. Hierf�r ist der Dispatcher zust�ndig.
Wichtig hierbei ist, dass der Compiler keine eigenen,
unkontrollierbaren Sicherungsversuche vornimmt, wie es
�blicherweise der Fall ist. Um dies zu verhindern wird die 
\gls{ISR} des Dispatchers f�r den AVR-Compiler mit dem Attribut 
\texttt{ISR\_NAKED} versehen.

Der erste Schritt des Dispatchers besteht darin, s�mtliche
CPU-Register auf dem Stack der aktuellen Task zu sichern.
Die Reihenfolge wie die Register auf den Stack gepushed
werden ist dabei nicht ma�gebend, jedoch muss beim
Widerherstellen der Task in umgekehrter Reihenfolge
vorgegangen werden. Nach der Sicherung wird der aktuelle
Stackpointer im \gls{TCB} \todo{verweis} abgelegt.

Der Dispatcher kann in zwei unterschiedlichen Situationen
aufgerufen werden: Entweder die Zeitscheibe der Task ist
abgelaufen und ein Dispatcher-Timer Interrupt wird ausgel�st
oder die Task gibt die CPU freiwillig ab\todo{s. yield}. Beide F�lle m�ssen
ab dieser Stelle unterschiedlich behandelt werden. Wurde 
der Dispatcher vom Timer ausgef�hrt (\texttt{hardwareISR = 1}),
muss die Systemzeit
erh�ht werden und gleichzeitig alle bis zu diesem Zeitpunkt
schlafenden Tasks aufgeweckt werden. Im anderen Fall 
(\texttt{hardwareISR = 0}) darf dieser Schritt nicht ausgef�hrt
werden. Weiterhin muss unterschieden werden, in welchem Zustand
sich die Task gerade befindet. Wurde sie blockiert
(\texttt{state = WAITING}), muss nichts weiter unternommen werden.
Ist sie jedoch noch aktiv (\texttt{state = RUNNING}) muss ihr 
Zustand auf \texttt{state = READY} gesetzt und im Scheduler
neu eingereiht werden.\todo{verweis scheduler}

Bevor der Scheduler aufgerufen wird, �berpr�ft der Dispatcher
mittels Magicnumber den Stack auf einen �berlauf. Sollte dies
der Fall sein, wird ein \texttt{kernelPanic}-Fehler ausgel�st. \todo{verweis exceptions}

\section{Scheduler}
Abh�ngig vom ausgew�hlten Algorithmus w�hlt der Scheduler
die n�chste auszuf�hrende Task aus. Au�erdem verwaltet er
die anstehenden, lauff�higen Tasks. Zum Systemstart kann
der gew�nschte Scheduleralgorithmus ausgew�hlt werden. \todo{verweis systemstart}
Zudem kann er sogar in Laufzeit ge�ndert werden. Dazu
muss einfach die Initialisierungsfunktion des neuen 
Schedulers aufgerufen werden. Darin werden im System verwendete
Funktionspointer auf die jeweiligen Algorithmenfunktionen gesetzt.
 Daher ben�tigt jede Implementierung
eines Scheduleralgorithmus folgende Funktionen:
\begin{itemize}
\item \texttt{void scheduler\_NextTask(void)}
\item \texttt{void scheduler\_enqueueTask(task)}
\item \texttt{void scheduler\_blockedByRessourceRequest(ressource)}
\item \texttt{void scheduler\_ressourceReleased(ressource)}
\end{itemize}

Die Funktion \texttt{scheduler\_NextTask} wird bei jedem 
Dispatcherdurchlauf aufgerufen. Sie w�hlt entsprechend dem
Algorithmus die n�chste Task aus der Menge der lauff�higen
Tasks aus.

\texttt{scheduler\_enqueueTask} wird dann ben�tigt, wenn eine
Task in den Zustand \texttt{READY} wechselt, unabh�ngig davon,
 welcher Zustand davor herrschte. So findet die Funktion u.a. beim
Verdr�ngen einer Task im Dispatcher oder beim Aufwecken im 
Zeiten-Modul \todo{verweis} Anwendung.

Die Funktionen \texttt{scheduler\_blockedByRessourceRequest} und 
\texttt{scheduler\_ressourceReleased} werden nur von wenigen 
Scheduleralgorithmen ben�tigt. Im XX\todo{name?}OS sind sie
ausschlie�lich im Priorit�tenvererbungs-Algorithmus (PIP) zu 
finden. \todo{verweis PIP} Sollten sie vom Algorithmus nicht
verwendet werden m�ssen sie dennoch mit einem leeren K�rper
implementiert werden.

\subsection{Scheduleralgorithmen}
XX\todo{name?}OS unterst�tzt folgende Algorithmen:
\begin{itemize}
\item einfaches Round-Robin
\item priorit�tengesteuertes Round-Robin
\item Priorit�tenvererbung (PIP) \todo{verweis}
\end{itemize}
\textit{Mit Ausnahme von PIP soll nicht n�her auf die
Implementierung der einzelnen Algorithmen eingegangen, da 
diese weitestgehend bekannt sind.}

\subsection{Idle-Task}
Ist keine der Tasks zum gleichen Zeitpunkt im lauff�higen 
Zustand muss es dennoch eine M�glichkeit
geben, dass der Dispatcher immerfort aufgerufen wird, damit
neue lauff�hige Tasks ausgew�hlt werden k�nnen. Die einfachste
Umsetzung hierf�r ist die Initialisierung einer sogenannten 
Idle-Task, welche in einer Dauerschleife "freiwillig" ihre Zeitscheibe beendet und dadurch den Dispatcher aufruft.
Wichtig hierbei ist, dass die Idle-Task nie in einen blockierten
Zustand versetzt werden darf und ihre Priorit�t auf das Maximum
gesetzt wird. Letzteres verhindert, dass die Idle-Task ausgef�hrt
wird, obwohl andere Tasks lauff�hig sind.
\begin{lstlisting}[language=C, label=lst:idleYield, caption=Idle-Task und yield]
void idleTask(){
	while(1){
		task_yield();
	}
}
void task_yield(){
	disableInterrupts();
	hardwareISR = 0;
	DISPISRVEC();
}
\end{lstlisting}

\section{Synchronisation}
F�r die Interaktion zwischen Tasks und die Verwaltung von 
Ressourcen werden in Betriebssystemen Synchronisationsmechanismen 
verwendet. Ihre Funktionalit�t baut im Grunde auf den kritischen
Abschnitt auf, welcher Interrupts sperrt. XX\todo{name?}OS 
unterst�tzt neben dem kritischen Abschnitt auch Signale und
Semaphore (und damit auch Mutexe). 

\subsection{Semaphore}


\subsection{Signale}


\section{Zeiten}
Zur Verz�gerung einer Task f�r eine bestimmte Zeit wird das
Zeiten-Modul ben�tigt. Damit ist es m�glich, Tasks in einen
schlafenden Zustand (bzw. \texttt{WAITING}) in Abh�ngigkeit
der Zeit zu versetzen. In der sp�teren Anwendung ist das z.B.
f�r periodische Aktionen sehr hilfreich.

Soll eine Task schlafen gelegt werden, wird im \gls{TCB} 
(wie bereits erw�hnt) ihr Status auf \texttt{WAITING} gesetzt.
Zudem wird im Attribut \texttt{waitUntil} eingetragen bis zu 
welchem absoluten Systemzeitpunkt sie in diesem Zustand verweilen 
soll. Danach wird sie in einer verketteten Liste \todo{verweis} 
aufsteigend nach \texttt{waitUntil} einsortiert. Die Sortierung
vereinfacht das sp�tere Aufwecken.

Wird nun der Dispatcher \todo{verweis} vom Dispatcher-Timer
ausgel�st wird die Systemzeit inkrementiert. Aufgrund der �nderung
muss �berpr�ft werden, ob es Tasks gibt, die bis zu diesem
Zeitpunkt ins Schlafen versetzt wurden. Dazu wird �ber die
o.g. Liste iteriert, die entsprechenden Tasks aufgeweckt
und zum Einsortieren in die Menge der lauff�higen Tasks an den
Scheduler �bergeben.

\section{Collections}

\section{Exceptions}

\include{Inhalt/umsetzung}
\chapter{Tests}
\label{chap:tests}
Um die einzelnen Module des Betriebssystems unabh�ngig testen
zu k�nnen wurde f�r den AVR-Simulator eine eigene Testbench
entwickelt. Mit ihr ist es m�glich s�mtliche Szenarien 
durchzutesten und auf Richtigkeit zu �berpr�fen.

\section{Neustart}
F�r jeden Test sollten dabei eine definierte Umgebung vorliegen,
welche unabh�ngig von zuvor ausgef�hrten Tests sein muss.
Dadurch kann eine flexible Wiederholbarkeit erreicht werden.
Damit alle Tests automatisch ausgef�hrt werden k�nnen wird
ein Betriebssystemneustart ben�tigt. Dazu wird eine Interruptsperre
verh�ngt, der Dispatcher-Timer gestoppt, der Stack geleert, alle
CPU-Register zur�ckgesetzt und schlie�lich zur Programm-Adresse
0 gesprungen.

\begin{lstlisting}[language=C, label=lst:reboot, caption=Neustart]
void beerOS_reboot(){
	disableInterrupts();
	stopDispatcherTimer();
	//clear main stack
	while(mainSP <= RAMEND){
		*mainSP = 0;
		mainSP++;
	}
	mainCalled--;
	SP = 0;
	asm volatile(
		"CLR R0\n\t"
		[...] //clear registers
		"BCLR 7\n\t"
		);
	asm volatile ("jmp 0");
}
\end{lstlisting}
\clearpage

\section{\texttt{.noinit}-Variablen}
Zum Schutz vor einem fehlerhaften mehrfachen Aufrufen der 
\texttt{main}-Funktion wurde ein Z�hler implementiert. Da bei einem
Neustart ein solcher Aufruf gewollt ist wird die Z�hlervariable
\texttt{mainCalled} dekrementiert. Vor dem Start der gewrappten
Startfunktion \texttt{run} wird abgefragt, ob der \texttt{main}-Aufruf
gewollt ist.
\begin{lstlisting}[language=C, label=lst:main, caption=\texttt{main}-Funktion]
int main(void){
	if(mainCalled){
		kernelPanic();
	}
	mainCalled++;
	return run();
}
\end{lstlisting}

Globale Variablen werden vom Compiler in der Weise behandelt, dass
sie beim Systemstart (Durchlauf der Programmadresse 0) 
initialisiert werden. Jedoch ist dieser Automatismus f�r zwei F�lle
nicht gew�nscht:
\begin{enumerate}
\item W�rde die o.g. Variable \texttt{mainCalled} bei jedem Aufruf
der Programmadresse 0 initialisiert bzw. zur�ck gesetzt werden, so w�re die Bedingung in Zeile 2 (Code \ref{lst:main}) nie g�ltig, obwohl
ein Fehlerfall vorl�ge.
\item Auch die Funktionspointer-Variable 
\texttt{void (*initNextTest)(void)}, in der der n�chste 
auszuf�hrende Test gespeichert wird, w�rde zur�ckgesetzt werden.
Automatisiertes Testen w�re somit unm�glich.
\end{enumerate}

Abhilfe hier schafft das Attribut \texttt{(section (".noinit"))}
welches dem Compiler vermittelt, dass diese Variablen zwar 
angelegt, jedoch nicht initialisiert werden sollen.
\footnote{In gewisser Weise werden sie beim ersten Start des 
AVR-Simulators auf die angegebenen Werte initialisiert. Daher 
entsteht beim Compilieren eine Warnmeldung.}

\begin{lstlisting}[language=C, label=lst:noinit, caption=\texttt{noinit}-Variablen]
void (*initNextTest)(void) __attribute__ ((section (".noinit"))) = &initSemaphoreTest;
uint8_t mainCalled __attribute__ ((section (".noinit"))) = 0;
\end{lstlisting}


% ==========================================================================
% Anhang
% ==========================================================================
% \appendix
% \chapter{Erste �berschrift im Anhang}
\label{kap:Anhang:Kapitel}

\section{Unter�berschrift im Anhang}
\label{kap:Anhang:Kapitel:Section}
Lorem ipsum dolor sit amet, consetetur sadipscing elitr, sed diam nonumy eirmod tempor invidunt ut labore et dolore magna aliquyam erat, sed diam voluptua. At vero eos et accusam et justo duo dolores et ea rebum. Stet clita kasd gubergren, no sea takimata sanctus est Lorem ipsum dolor sit amet. Lorem ipsum dolor sit amet, consetetur sadipscing elitr, sed diam nonumy eirmod tempor invidunt ut labore et dolore magna aliquyam erat, sed diam voluptua. At vero eos et accusam et justo duo dolores et ea rebum. Stet clita kasd gubergren, no sea takimata sanctus est Lorem ipsum dolor sit amet.

\chapter{Zweite �berschrift im Anhang}
\label{kap:Anhang:Kapitel2}
Text Text Text Text Text Text Text Text Text Text Text Text Text Text Text Text Text Text Text Text Text Text Text Text Text Text Text Text Text Text Text Text Text Text Text Text Text Text Text Text Text Text Text Text Text Text Text Text Text Text Text Text Text Text Text Text Text Text Text Text Text Text Text Text Text Text Text Text Text Text Text Text Text Text Text Text Text Text Text Text Text Text Text Text Text Text Text Text Text Text.

\section{Nochmal eine Unter�berschrift im Anhang}
\label{kap:Anhang:Kapitel2:Section}
Hier ist ein text

\subsection{eine UnterUnter�berschrift im Anhang}
\label{kap:Anhang:Kapitel2:Section:Subsection}
Hier ist ein text

% \cleardoublepage

% ==========================================================================
%	Literatur- Abbildung- und Abk�rzungsverzeichnis
% ==========================================================================
%Abk�rzungen ausgeben
% \printglossary[type=\acronymtype,title=Abk�rzungsverzeichnis,toctitle=Abk�rzungsverzeichnis,style=myacro]

%Symbole ausgeben
% \printglossary[type=symbolslist,title=Symbolverzeichnis,toctitle=Symbolverzeichnis,style=mysombol]

% \listoftables					% Tabellenverzeichnis

% \listoffigures				% Abbildungsverzeichnis

% \begin{RaggedRight} %Darstellung des Literaturverzeichnisses im Rausatz wegen URLs
% \renewcommand{\bibname}{Quellenverzeichnis}	% Literaturverzeichnis umbenennen
% \bibliography{Dokumentation}    %Dateiname der .bib Datei
% \end{RaggedRight}
\clearpage

% ==========================================================================
%	ToDo-Liste (kann auskommentiert werden, wenn nicht ben�tigt)
% ==========================================================================
% \listoftodos[To-do Liste]
% \todototoc

\end{document}
