\chapter{System�bersicht}

F�r ein minimalistisches Betriebssystem werden einige Grundbausteine
ben�tigt. Zun�chst muss die Hardware bzw. der Chip initialisiert
werden, indem verschiedene Zeitquellen, Timer und die globale
Interruptverwaltung konfiguriert werden. \todo{verweis} Bevor
das eigentliche Betriebssystem gestartet werden kann, m�ssen zudem
diverse Strukturen f�r die Tasks aufgestellt werden, sodass diese
unabh�ngig voneinander lauff�hig sind. \todo{verweis}
Zum Starten des Systems wird mittels eines modifizierten Aufrufs
der Dispatcher erstmalig durchlaufen und die erste Task zum Ausf�hren
vorbereitet. Dabei wird auch der Scheduler aufgerufen, der, abh�ngig vom
ausgew�hlten Algorithmus, die n�chste Task bestimmt.\todo{verweis}

Die soeben genannten Module w�rden f�r ein multitasking-f�higes
Betriebssystem bereits ausreichen. Jedoch w�re sein Funktionsumfang
sehr beschr�nkt. Bislang besteht noch keine M�glichkeit einer sicheren
Interaktion zwischen zwei oder mehreren Tasks. Hierf�r werden 
z.B. kritische Bereiche oder Signale verwendet, welche das Modul
Synchronisation bereitstellt. Die Anwendung von Wartezeiten innerhalb
Tasks setzt eine Systemzeit vorraus. Das Modul Zeiten \todo{verweis} verwendet hierf�r Trigger der Hardware.
Eine gro�e Erleichterung im Entwicklungsprozess (und auch 
im sp�teren Anwendungsfall) ist
die Verwendung von Collections, welche Datenpakete hantierbar 
macht. Gleichzeitig verbessert sich die Lesbarkeit des
Programmcodes.

\section{Hardware}
Als Zielplattform wurde die weit verbreitete AVR-Serie von Atmel
verwendet. Jedoch wurde w�hrend der Umsetzung darauf geachtet, 
dass bei einem Plattformwechsel m�glichst wenig Code angepasst
werden muss. Als Entwicklungstool wurde der AVR-Simulator verwendet,
welcher innherhalb des AVR-Studios von Atmel bereitgestellt wird.
Als Zielbaustein wurde der ATxmega128A1 gew�hlt.

\subsection{Zeitquellen}
Zuallererst wird der interne Oszillator auf eine Systemfrequenz von $f_{S} = 32 \text{MHz}$ konfiguriert. Dadurch k�nnen durch
nachgeschaltete Teiler diverse Frequenzen an weiterer
Peripherie erzeugt werden \todo{verweis}.

\subsection{Dispatcher Timer}
Damit nach Ablauf der Zeitscheibe einer Task der Dispatcher 
aufgerufen wird, muss ein Dispatcher Timer initialisiert werden.
Dieser l�st, unabh�ngig vom Ausf�hrungszustand\todo{verweis}
\footnote{Es sei denn, die Task befindet sich in einem kritischen 
(Interrupt gesch�tzten) Abschnitt} der aktuellen Task, einen 
Interrupt aus, in dessen Service Routine der Dispatcher aufgerufen
wird.



\section{Strukturen}

\section{Dispatcher}

\section{Scheduler}

\section{Synchronisation}

\section{Zeiten}

\section{Collections}