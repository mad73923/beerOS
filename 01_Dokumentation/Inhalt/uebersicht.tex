\chapter{System�bersicht}

F�r ein minimalistisches Betriebssystem werden einige Grundbausteine
ben�tigt. Zun�chst muss die Hardware bzw. der Chip initialisiert
werden, indem verschiedene Zeitquellen, Timer und die globale
Interruptverwaltung konfiguriert werden. \todo{verweis} Bevor
das eigentliche Betriebssystem gestartet werden kann, m�ssen zudem
diverse Strukturen f�r die Tasks aufgestellt werden, sodass diese
unabh�ngig voneinander lauff�hig sind. \todo{verweis}
Zum Starten des Systems wird mittels eines modifizierten Aufrufs
der Dispatcher erstmalig durchlaufen und die erste Task zum Ausf�hren
vorbereitet. Dabei wird auch der Scheduler aufgerufen, der, abh�ngig vom
ausgew�hlten Algorithmus, die n�chste Task bestimmt.\todo{verweis}

Die soeben genannten Module w�rden f�r ein multitasking-f�higes
Betriebssystem bereits ausreichen. Jedoch w�re sein Funktionsumfang
sehr beschr�nkt. Bislang besteht noch keine M�glichkeit einer sicheren
Interaktion zwischen zwei oder mehreren Tasks. Hierf�r werden 
z.B. kritische Bereiche oder Signale ben�tigt, welche das Modul
Synchronisation bereitstellt. Die Anwendung von Wartezeiten innerhalb
Tasks setzt eine Systemzeit vorraus. Das Modul Zeiten \todo{verweis} verwendet hierf�r Trigger der Hardware.
Eine gro�e Erleichterung im Entwicklungsprozess (und auch 
im sp�teren Anwendungsfall) ist
die Verwendung von Collections, welche Datenpakete hantierbar 
macht. Gleichzeitig verbessert sich die Lesbarkeit des
Programmcodes.

\section{Hardware}
Als Zielplattform wurde die weit verbreitete AVR-Serie von Atmel
verwendet. Jedoch wurde w�hrend der Umsetzung darauf geachtet, 
dass bei einem Plattformwechsel m�glichst wenig Code angepasst
werden muss. Als Entwicklungstool wurde der AVR-Simulator verwendet,
welcher innherhalb des AVR-Studios von Atmel bereitgestellt wird.
Als Zielbaustein wurde der \mbox{ATxmega128A1} gew�hlt.

Die minimale Vorraussetzung an die Hardware ist ein Timer
und ein Interruptsystem.
Die Periode des Timers muss bekannt oder definierbar sein.
Sind (wie im Fall vom \mbox{ATxmega128A1}) weitere Timer
verf�gbar, k�nnen diese vom Entwickler frei verwendet werden, 
unter der Bedingung, dass der Systemtimer nicht beeinflusst 
wird.

\subsection{Zeitquellen}
Zuallererst wird der interne Oszillator auf eine Systemfrequenz von $f_{S} = 32 \text{MHz}$ konfiguriert. Dadurch k�nnen durch
nachgeschaltete Teiler diverse Frequenzen an weiterer
Peripherie erzeugt werden \todo{verweis}.

\subsection{Dispatcher Timer}
Damit nach Ablauf der Zeitscheibe einer Task der Dispatcher 
aufgerufen wird, muss ein Dispatcher Timer initialisiert werden.
Dieser l�st, unabh�ngig vom Ausf�hrungszustand\todo{verweis}
\footnote{Es sei denn, die Task befindet sich in einem kritischen 
(Interrupt gesperrten) Abschnitt} der aktuellen Task, einen 
Interrupt aus, in dessen Service Routine der Dispatcher aufgerufen
wird. Au�erdem dient der Timer zur Berechnung der aktuellen
Systemzeit.

\begin{lstlisting}[language=C, label=lst:DispatcherTimer, caption=Initialisierung Dispatcher Timer]
void initDispatcherTimer(){
	TCF0.CTRLB = TC_WGMODE_NORMAL_gc;
	TCF0.PER = 0x7D00;
	TCF0.INTCTRLA = TC_OVFINTLVL_HI_gc;
}
\end{lstlisting}

In dieser Konfiguration z�hlt der Timer aufw�rts bis zum Wert der
Periode (\texttt{TCF0.PER}) und wird auf 0 zur�ckgesetzt. 
Gleichzeitig l�st er beim Zur�cksetzen einen �berlauf-Interrupt aus.

Wird als Zeitquelle die Systemfrequenz $f_{S}$ ohne Vorteiler
gew�hlt, so ergibt sich folgendes Intervall:
\begin{equation}
\begin{split}
t_{per} = \frac{7\text{D}00_{16}}{32\text{MHz}} = 1\text{ms}
 \end{split}
 \label{eq:intervall}
\end{equation}

In der Annahme, dass keine kritische Abschnitte verwendet werden
wird somit in Abst�nden von $t_{per}$ ein Kontextwechsel herbeigef�hrt. Sollten kritische Abschnitte Anwendung finden liegt
es am Entwickler, diese m�glichst kurz zu halten. Solange sie
deutlich unter $t_{per}$ liegen gibt es keine Probleme. Bedenklich 
wird es, wenn die Ausf�hrung des kritischen Bereichs l�nger als 
$t_{per}$ dauert,
da hierbei ein oder im schlimmsten Fall gleich mehrere 
Interruptsignale verloren gehen und dadurch u.a. die Systemzeit
nicht mehr pr�zise ist.

\subsection{Interrupts}
Auch die Konfiguration des Interruptsystems ist nicht besonders
aufw�ndig. Auf dem \mbox{ATxmega128A1} muss jediglich das
dem Dispatcher Timer entsprechende Interuptlevel scharf gestellt
und das globale Interruptenable gesetzt werden.

Da die Funktionen zum L�schen oder Setzen des Interrupts auch
mit dem Betreten oder Verlassen eines kritischen Abschnitts
gleichzusetzen sind, werden sie hardwareunabh�ngig exportiert.
Wichtig hierbei ist nur, dass die Funktionen inline aufgerufen
werden, sodass im Anwendungsfall der Stack nicht manipuliert wird.
Dies wird beim AVR-Compiler durch das Attribut 
\texttt{always\_inline} verwirklicht.
\begin{lstlisting}[language=C, label=lst:interruptDisEnable, caption=Export der Interruptfunktionen]
static void __attribute__((always_inline)) enableInterrupts(){
	sei();
}

static void __attribute__((always_inline)) disableInterrupts(){
	cli();
}
\end{lstlisting}

\section{Strukturen}

\section{Dispatcher}

\section{Scheduler}

\section{Synchronisation}

\section{Zeiten}

\section{Collections}