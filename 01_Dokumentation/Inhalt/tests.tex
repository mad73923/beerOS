\chapter{Tests}
\label{chap:tests}
Um die einzelnen Module des Betriebssystems unabh�ngig testen
zu k�nnen wurde f�r den AVR-Simulator eine eigene Testbench
entwickelt. Mit ihr ist es m�glich s�mtliche Szenarien 
durchzutesten und auf Richtigkeit zu �berpr�fen.

\section{Neustart}
F�r jeden Test sollten dabei eine definierte Umgebung vorliegen,
welche unabh�ngig von zuvor ausgef�hrten Tests sein muss.
Dadurch kann eine flexible Wiederholbarkeit erreicht werden.
Damit alle Tests automatisch ausgef�hrt werden k�nnen wird
ein Betriebssystemneustart ben�tigt. Dazu wird eine Interruptsperre
verh�ngt, der Dispatcher-Timer gestoppt, der Stack geleert, alle
CPU-Register zur�ckgesetzt und schlie�lich zur Programm-Adresse
0 gesprungen.

\begin{lstlisting}[language=C, label=lst:reboot, caption=Neustart]
void beerOS_reboot(){
	disableInterrupts();
	stopDispatcherTimer();
	//clear main stack
	while(mainSP <= RAMEND){
		*mainSP = 0;
		mainSP++;
	}
	mainCalled--;
	SP = 0;
	asm volatile(
		"CLR R0\n\t"
		[...] //clear registers
		"BCLR 7\n\t"
		);
	asm volatile ("jmp 0");
}
\end{lstlisting}
\clearpage

\section{\texttt{.noinit}-Variablen}
Zum Schutz vor einem fehlerhaften mehrfachen Aufrufen der 
\texttt{main}-Funktion wurde ein Z�hler implementiert. Da bei einem
Neustart ein solcher Aufruf gewollt ist wird die Z�hlervariable
\texttt{mainCalled} dekrementiert. Vor dem Start der gewrappten
Startfunktion \texttt{run} wird abgefragt, ob der \texttt{main}-Aufruf
gewollt ist.
\begin{lstlisting}[language=C, label=lst:main, caption=\texttt{main}-Funktion]
int main(void){
	if(mainCalled){
		kernelPanic();
	}
	mainCalled++;
	return run();
}
\end{lstlisting}

Globale Variablen werden vom Compiler in der Weise behandelt, dass
sie beim Systemstart (Durchlauf der Programmadresse 0) 
initialisiert werden. Jedoch ist dieser Automatismus f�r zwei F�lle
nicht gew�nscht:
\begin{enumerate}
\item W�rde die o.g. Variable \texttt{mainCalled} bei jedem Aufruf
der Programmadresse 0 initialisiert bzw. zur�ck gesetzt werden, so w�re die Bedingung in Zeile 2 (Code \ref{lst:main}) nie g�ltig, obwohl
ein Fehlerfall vorl�ge.
\item Auch die Funktionspointer-Variable 
\texttt{void (*initNextTest)(void)}, in der der n�chste 
auszuf�hrende Test gespeichert wird, w�rde zur�ckgesetzt werden.
Automatisiertes Testen w�re somit unm�glich.
\end{enumerate}

Abhilfe hier schafft das Attribut \texttt{(section (".noinit"))}
welches dem Compiler vermittelt, dass diese Variablen zwar 
angelegt, jedoch nicht initialisiert werden sollen.
\footnote{In gewisser Weise werden sie beim ersten Start des 
AVR-Simulators auf die angegebenen Werte initialisiert. Daher 
entsteht beim Compilieren eine Warnmeldung.}

\begin{lstlisting}[language=C, label=lst:noinit, caption=\texttt{noinit}-Variablen]
void (*initNextTest)(void) __attribute__ ((section (".noinit"))) = &initSemaphoreTest;
uint8_t mainCalled __attribute__ ((section (".noinit"))) = 0;
\end{lstlisting}
