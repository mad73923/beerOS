\chapter{Features}

\section{Speicherverwaltung}

\section{Scheduler: Priorit�tenvererbung}
Bei der Umsetzung des Priority-Inheritance-Protocols (PIP) wurde 
ein flexibler, selbst-lernender Ansatz erarbeitet. Dabei erweitert
er im Grunde den priorit�tengesteuerten Round-Robin Algorithmus
um zwei Punkte:

\begin{enumerate}
\item Jede Ressource (darunter fallen auch Teile des 
Synchronisations-Moduls) erh�lt jeweils eine eigene Liste.
Darin wird vermerkt, welche Tasks die Ressource jemals freigegeben
haben.
\item Wird nun eine Task durch eine Ressourcenanforderung 
blockiert, �berpr�ft der Algorithmus, ob eine Task aus der
in 1. genannten Liste f�r eine Priorit�tenvererbung in Frage
kommt.
\end{enumerate}

Diese Methode f�hrt dazu, dass nachdem alle Tasks die ben�tigten
Ressourcen mindestens einmal verwendet haben es zu keiner
Priorit�teninversion kommen kann. Soll diese Bedingung gleich
zu Beginn gelten, so kann optional die Freigabeliste bereits
vom Entwickler vorgegeben werden.

\subsection{Blockierung durch Ressourcenanforderung}
Im Fall einer Blockierung wird �ber die o.g. Freigabeliste 
iteriert. Als Bedingung, dass eine Priorit�tenvererbung stattfindet
muss die bevorzugte Task eine niedrigere Priorit�t haben und
lauff�hig sein (\texttt{state = READY}). Ist beides erf�llt,
wird der Task tempor�r die Priorit�t der blockierten Task 
zugewiesen und in die Menge der lauff�higen Tasks neu einsortiert.

\begin{lstlisting}[language=C, label=lst:blockedTask, caption=Verwaltung einer blockierten Ressourcenanforderung]
void prioInheritance_blockedByRessourceRequest(LinkedList* resFreedBy){
	uint8_t length = linkedList_length(resFreedBy);
	taskControlBlock* nextTask;
	if(length){
		for(uint16_t i = 0; i < length; i++){
			linkedList_get(resFreedBy, i, &nextTask);
			if(nextTask->state == READY && nextTask->prio > currentTask->prio){
				nextTask->prio = currentTask->prio;
				queue_removeItem(&prioQueue[nextTask->tmpPrio], nextTask);
				queue_push(&prioQueue[nextTask->prio], nextTask);
				break;
			}
		}
	}
}
\end{lstlisting}

\subsection{Freigabe einer Ressource}
Wird bei PIP eine Ressource freigegeben, wird zun�chst �berpr�ft,
ob die freigebende Task bereits im der Freigabeliste enthalten ist.
Ist dies nicht der Fall, wird sie eingetragen. Wichtiger 
jedoch muss erkannt werden, ob eine Priorit�tenvererbung 
stattgefunden hat. Falls ja, wird die Priorit�t auf den 
urspr�nglichen Wert (\texttt{tmpPrio}) zur�ckgesetzt und die Task
verdr�ngt, da sie nun die Ressource nicht mehr f�r sich in
Anspruch nimmt\todo{verweis zeile 7}.

\begin{lstlisting}[language=C, label=lst:freedTask, caption=Ressource wird freigegeben]
void prioInheritance_ressourceReleased(LinkedList* resFreedBy){
	if(!linkedList_contains(resFreedBy, currentTask)){
		linkedList_append(resFreedBy, currentTask);
	}
	if(currentTask->tmpPrio != currentTask->prio){
		currentTask->prio = currentTask->tmpPrio;
		task_yield();
	}
}
\end{lstlisting}
